\chapter{New GEMs in CMS}
\epigraph{\itshape All have their worth… and each contributes to the worth of the others}{J.R.R. Tolkien, \textit{The Silmarillion} }

%
The Gas Electron Multipliers are novel gas chamber detectors being added to the CMS experiment in order to improve the triggering capabilities of the Muon System.


\section{GE11 Detector}
%

\section{GE11 Electronics}

\subsection{Front-end Electronics}
%

\subsubsection{GEB: GEM Electronics Board}
%

\subsubsection{VFAT3}
%

\subsubsection{OptoHybrid}
%

\subsection{Back-end Electronics}
%
The goal of the back-end electronic system is to interface the GEM detectors with CMS DAQ (Data-Acquisition), TTC (Trigger Timing and Control), and Trigger. This is accomplished using Multi-Gbit/s links that adhere to a standard approach from the telecom industry called $\mu$TCA. This allows for high throughput (2 Tbit/s) and high availability ($~10^{-5}$ low chance of being interrupted). $\mu$TCA is now the standard for all CMS upgrades.

The $\mu$TCA crate used by the GE11 is VadaTech VT892, an over view of which is given in figure (BLANK). This crate houses two slots for MCH modules and 12 slots for Advanced Mezzanine Cards (AMCs). The first MCH slot is occupied by blah, blah. The second MCH slot houses the AMC13 module.

\subsubsection{AMC13}

The AMC13 is a MCH module developed by Boston University. It is the CMS standard module for providing an interface between the $\mu$TCA crate to CMS DAQ and TTC.

\subsubsection{AMC: Advanced Mezzanine Card}

GE11 uses two types of AMCs: CTP7 for point 5 operations and GLIB for test stand operations.

The Calorimeter Trigger Processor Card (CTP7) is an AMC developed by the University of Wisconsin. 

\section{Data-acquisition}
%